\clearpage

\section{Simulation Analysis}
\label{sec:simulation}
\vspace{-0.5cm}
In this section we shall describe the simulation of the circuit presented in figure \ref{fig:Principal}. This circuit was translated into \emph{ngspice}
source code and simulated as follows.
Firstly, we did transient analysis in order to study the output of the circuit and so we could check whether or
not it had any visible distortion. Having confirmed the abscence of said distortion, we proceeded to the frequency analysis,
the main focus of this lab assignment.
As such, we decided to compute both the input and output impedances of the circuit used, as they are critical for
understanding the behaviour of the circuit. The results obtained are shown in the table below. Keep in mind that these values were calculated
at the central frequency, which we shall present later in the analysis.

\vspace{-0.65cm}
\begin{table}[h]\label{tab:SimZ}
    \centering
    \begin{tabular}{|c|c|}
      \hline
      {\bf Name} & {\bf Value [$\Omega$]} \\ \hline
      \input{../sim/SimZ_tab.tex}
    \end{tabular}
    \vspace{-2.5mm}
    \caption{\emph{Ngspice}'s input and output impedances' absolute values.}
\end{table}

\vspace{-0.45cm}
The first and second lines of the table contain, respectively, the input and output impedances' absolute values.
Generally, when designing a circuit such as this one, we want it to have a high input impedance,
so that the input signal isn’t degredated, and a low output impedance, which makes it so that the
output signal, which is delivered to the load, isn’t degredated as well. As we
can see from the table above, the values obtained are far from ideal. The input impedance is far from infinite and the output impedance is also
distant from zero. Nevertheless, we figured the results were good enough despite this fact, so we didn't bother trying to increase the input
impedance nor decrease the output impedance any further. However, this is a fact that still needs to be kept in mind if we inted to connect this circuit
to other circuits. We would also like to point out that, in order to compute the output impedance, we had to utilize a different \emph{ngspice} 
script, in which we switched off the input voltage source, $V_{in}$ , in figure \ref{fig:Principal}, and connected a test voltage source to
the output of the circuit. The output impedance is then given by the quotient
of this test voltage by the current flowing through the test voltage itself.
Following the previous frequency analysis, we obtained the graph shown in figure \ref{fig:NgspiceGain_dB}, for the
gain in dB, and the graph shown in figure \ref{fig:NgspicePhase}, for the phase of the gain.

\vspace{-0.25cm}
\begin{figure}[h] \centering
  \includegraphics[scale=0.515]{sim_gain_db.pdf}
  \caption{\emph{Ngspice} gain in dB.}
  \label{fig:NgspiceGain_dB}
\end{figure}

\begin{figure}[h] \centering
  \includegraphics[scale=0.515]{sim_phase.pdf}
  \caption{\emph{Ngspice} gain's phase.}
  \label{fig:NgspicePhase}
\end{figure}

Utilizing the graph shown in figure \ref{fig:NgspiceGain_dB}, we were able to compute the maximum gain of the circuit, the lower
and upper cut-off frequencies, as well as the central frequency, defined as
the geometric mean of the lower and upper cut-off frequencies. The values obtained are presented in the following tables.

\begin{table}[h]
  \parbox{.45\linewidth}{
    \centering
    \begin{tabular}{|c|c|}
      \hline
      {\bf Name} & {\bf Value [Hz]} \\ \hline
      \input{../sim/SimFreqs_tab.tex}
    \end{tabular}
    \caption{\emph{Ngspice}'s cut-off frequencies and central frequency.}
  }
  \hfill
  \parbox{.45\linewidth}{
    \centering
    \begin{tabular}{|c|c|}
      \hline
      {\bf Name} & {\bf Value [dB or -]} \\ \hline
      \input{../sim/SimGain_tab.tex}
    \end{tabular}
    \caption{\emph{Ngspice}'s gain in dB and linear gain.}
  }
\end{table}

In designing this filter circuit we didn't really worry about its bandwidth as it wasn't necessary for what we wanted to achieve. Instead, we focused
on its gain, as well as its central frequency. We managed to achieve a gain which was really close to 40 dB, as was intended. The central frequency, however,
is relatively far away from the intended 1 kHz. Given the restrictions imposed by the professor in terms of available components, we had some 
trouble getting both the gain and central frequency to match the intended values. As such we decided to prioritize the gain of the filter
circuit and in doing such ended up sacrificing the central frequency. In hindsight, we realized this wasn't the most ideal, as our circuit was supposed
to be a filter circuit, which means we should have focused on its central frequency instead.

\clearpage

Anyways, moving on, utilizing these results we were able to compute the merit of our work, making use of the following formula.

\begin{equation}
M = \frac{1}{\text{Cost}\times \left(\text{voltage gain deviation} + \text{central frequency deviation} + 10^{-6}\right)}
\end{equation}

The result is presented in the table below, in which we also show the cost of our circuit.

\begin{table}[h]
  \centering
  \begin{tabular}{|l|r|}
    \hline
    {\bf Name} & {\bf Value [MU or -]} \\ \hline
    \input{../sim/SimMerit_tab.tex}
  \end{tabular}
  \caption{Cost and merit using \emph{Ngspice}'s results.}
  \label{tab:CostAndMerit}
\end{table}

The cost was computed using the following scheme.

\begin{itemize}
  \item Resistors $\rightarrow$ 1 MU per k$\Omega$;
  \item Capacitors $\rightarrow$ 1 MU per $\mu$F;
  \item Transistors $\rightarrow$ 0.1 MU per Transistor;
  \item Diode $\rightarrow$ 0.1 MU per Diode (used in OP-AMP).
\end{itemize}