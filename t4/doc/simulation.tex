\clearpage

\section{Simulation Analysis}
\label{sec:simulation}

In this section we shall describe the simulation of the circuit presented in figure \ref{fig:circuito}.
The models used for the npn and pnp transistors shown in figure \ref{fig:circuito} are, respectively, the BC557A and the BC547A,
both Philips' products.
As far as the actual simulation is concerned, firstly we did a transient analysis in order to study the output of the circuit and
so we could check whether or not the output had any visible distortion. Having confirmed the abscence of said distortion, we decided to print out the values
of the voltages in the transistors so we could verify that they were, in fact, in the forward active region (F.A.R). These values
are presented in the following tables.

\begin{table}[h]
  \parbox{.45\linewidth}{
    \centering
    \begin{tabular}{|c|c|}
      \hline
      {\bf Name} & {\bf Value [V]} \\ \hline
      \input{../sim/SimFARnpn_tab.tex}
    \end{tabular}
    \caption{Ngspice's npn transistor voltages.}
  }
  \hfill
  \parbox{.45\linewidth}{
    \centering
    \begin{tabular}{|c|c|}
      \hline
      {\bf Name} & {\bf Value [V]} \\ \hline
      \input{../sim/SimFARpnp_tab.tex}
    \end{tabular}
    \caption{Ngspice's pnp transistor voltages.}
  }
\end{table}

As can be seen, the transistors are indeed in the F.A.R, as $V_{CE} > V_{BE}$, for the npn transistor, and $V_{EC} > V_{EB}$, for the pnp transistor.
We can also see that $V_{BE}\approx V_{EB}\approx 0.7\hspace{1mm}V$. This value ($0.7\hspace{1mm}V$) is often used as an estimate of both $V_{BE}$ and $V_{EB}$,
for theoretical analysis, as it simplifies the calculations. We too shall use this approximation in the next section.

After this, we moved on to the frequency analysis. Before anything else, we decided to compute both the input and output impedances of the circuit used,
as they are critical for understanding the behaviour of the circuit. The results obtained are shown in the tables below.

\begin{table}[h]
  \parbox{.45\linewidth}{
    \centering
    \begin{tabular}{|c|c|}
      \hline
      {\bf Name} & {\bf Value [$k\Omega$]} \\ \hline
      \input{../sim/SimZi_tab.tex}
    \end{tabular}
    \caption{Ngspice's input impedance.}
  }
  \hfill
  \parbox{.45\linewidth}{
    \centering
    \begin{tabular}{|c|c|}
      \hline
      {\bf Name} & {\bf Value [$\Omega$]} \\ \hline
      \input{../sim/SimZo_tab.tex}
    \end{tabular}
    \caption{Ngspice's output impedance.}
  }
\end{table}

In the first line of both tables, the impedance is presented in its complex form (first number refers to the real part, second refers to the imaginary part).
The second line of both tables contains the impedances' absolute value.
A good amplifier is one that has a high input impedance, so that the input signal isn't degredated, a low output impedance, which makes it so that
the output signal, which is delivered to the load, isn't degredated as well, and a high gain. As we can see from the table above, the 
output impedance of our simulated amplifier is indeed low. The input impedance however, although much greater than the output impedance, could still be higher.
Nevertheless, we figured the results were good enough despite this fact, so we didn't bother trying to increase the input impedance any further.
We would also like to point out that, in order to compute the output impedance, we had to utilize a different \emph{ngspice} script, in which we switched 
off the input voltage source, $V_{S}$, in figure \ref{fig:circuito}, and connected a test voltage source to the output of the citcuit, instead of the load.
The output impedance is then given by the quotient of this test voltage by the current flowing through the test voltage itself.

Following the previous frequency analysis, we obtained the graph shown in figure \ref{fig:NgspiceGain_dB}, for the gain in dB.

\clearpage

\begin{figure}[h] \centering
  \includegraphics[scale=0.7]{spice_gain_db.pdf}
  \caption{Ngspice gain in dB.}
  \label{fig:NgspiceGain_dB}
\end{figure}

Utilizing this graph, we were able to compute the maximum gain of the amplifier. The lower and upper cut-off frequencies, as well as the bandwidth
of the amplifier, were calculated using the output signal.
The values obtained are presented in the following tables.

\begin{table}[h]
  \parbox{.45\linewidth}{
    \centering
    \begin{tabular}{|c|c|}
      \hline
      {\bf Name} & {\bf Value [Hz]} \\ \hline
      \input{../sim/SimFreqs_tab.tex}
    \end{tabular}
    \caption{Ngspice's cut-off frequencies and bandwidth.}
  }
  \hfill
  \parbox{.45\linewidth}{
    \centering
    \begin{tabular}{|c|c|}
      \hline
      {\bf Name} & {\bf Value [dB or -]} \\ \hline
      \input{../sim/SimGain_tab.tex}
    \end{tabular}
    \caption{Ngspice's gain.}
  }
\end{table}

As we can see, the simulated amplifier has a lower cut-off frequency close to $20\hspace{1mm}Hz$, which is really good, provided that the human 
ear cannot detect sounds below the $20\hspace{1mm}Hz$ mark. This also means that we are not wasting any part of the bandwidth, given that the upper cut-off
frequency is way above the limit of $20\hspace{1mm}kHz$, which is the maximum frequency the human ear can detect. With that said, one can make the argument
that we cannot hear sounds higher than $20\hspace{1mm}kHz$ and, as such, we really are wasting part of the bandwidth. However, it turned out the upper
cut-off frequency was generally very high, much higher than $20\hspace{1mm}kHz$, regardless of the amplifier's specs. So, since there is not much we can do 
about that, we instead focus on the minimization of the lower cut-off frequency.

Lastly, we obtained a gain of 37.68241 dB (76.58091 in a linear scale), which seems high enough.

Utilizing these results, we were able to compute the merit of our work, making use of the following formula.

\begin{equation}
  M=\dfrac{voltageGain\times bandwidth}{cost\times lowercutoffFreq}
\end{equation}

The result is presented in the table below, in which we also show the cost of our circuit.

\begin{table}[h]
  \centering
  \begin{tabular}{|l|r|}
    \hline
    {\bf Name} & {\bf Value [MU or -]} \\ \hline
    \input{../sim/SimMerit_tab.tex}
  \end{tabular}
  \caption{Cost and merit using Ngspice's results.}
  \label{tab:CostAndMerit}
\end{table}

The cost was computed using the following scheme.

\begin{itemize}
  \item Resistors $\rightarrow$ 1 MU per k$\Omega$;
  \item Capacitors $\rightarrow$ 1 MU per $\mu$F;
  \item Transistors $\rightarrow$ 0.1 MU per Transistor.
\end{itemize}

%\begin{figure}[h] \centering
%  \includegraphics[scale=0.5]{}
%  \caption{Voltages at the output of the envelope detetctor (blue) and voltage regulator circuits (red): detail in the interval [20, 60] ms.}
%  \label{fig:33}
%\end{figure}

%\begin{figure}[h] \centering
%  \includegraphics[scale=0.5]{}
%  \caption{Output AC component (V(out)-12).}
%  \label{fig:22}
%\end{figure}

%\begin{figure}[h] \centering
%  \includegraphics[scale=0.8]{}
%  \caption{Output AC component (V(out)-12): detail in the interval [20, 60] ms.}
%  \label{fig:44}
%\end{figure}
