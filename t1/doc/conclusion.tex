\section{Conclusion}
\label{sec:conclusion}

In this laboratory assignment, the objective of analysing the circuit presented
in figure \ref{fig:rc} has been achieved. The analysis was performed both theoretically
using the Octave maths tool and by circuit simulation using the Ngspice tool.
The simulation results matched the theoretical results precisely.
The reason for this perfect match is the fact that this is a
straightforward circuit containing only linear components, so the theoretical
and simulation models cannot differ. For more complex components, the
theoretical and simulation models could differ, but this is not the case in this
work. Even though the calculations were successful, we realised that we could have 
done a smarter choice of the reference node. For instance, if we had chosen node $4$ as the reference node, 
the equations would be a little simpler overall, since there would be more voltage values in the equations that would 
be equal to $0$. In addition, this choice of reference node would make us apply the supernode construct for the independent 
voltage source instead of the dependent one, making the equations easier once again, since there would be less diverging and 
converging currents to take into account.

\begin{thebibliography}{9}

    \bibitem{spicenotes}
    Phyllis R. Nelson.
    \textit{\href{https://www.cpp.edu/~prnelson/courses/ece220/220-spice-notes.pdf}{Introduction to \emph{spice} source files}}.
    \\\texttt{https://www.cpp.edu/ $\tilde{}$ prnelson/courses/ece220/220-spice-notes.pdf}


    \bibitem{octave}
    John W. Eaton, David Bateman, Søren Hauberg, Rik Wehbring.
    \textit{\href{https://octave.org/octave.pdf}{GNU Octave - Free Your Numbers}}.
    \\\texttt{https://octave.org/octave.pdf}

    \bibitem{libreoffice}
    LibreOffice Documentation Team.
    \textit{\href{https://documentation.libreoffice.org/assets/Uploads/Documentation/en/GS7.0/GS70-GettingStarted.pdf}{Getting Started Guide}}.
    \\\texttt{https://documentation.libreoffice.org/assets/Uploads/Documentation/en/
        GS7.0/GS70-GettingStarted.pdf}

    \bibitem{analysis}
    Mesh analysis and node analysis notes.
    \textit{\href{https://moodle.fct.unl.pt/pluginfile.php/167552/mod_resource/content/0/NotasTecnicas/Apontamento_tecnico_3_ebf.pdf}{Princípios fundamentais na análise de circuitos electrónicos}}.
    \\\texttt{https://moodle.fct.unl.pt/pluginfile.php/167552/mod$\_$resource/content/
        0/NotasTecnicas/Apontamento$\_$tecnico$\_$3$\_$ebf.pdf}

\end{thebibliography}
