\newpage
\section{Introduction}
\label{sec:introduction}

% state the learning objective 
The objective of this laboratory assignment is to study a circuit composed of
four elementary meshes where these can contain resistors, independent current or voltage sources
and dependent current or voltage sources as seen in figure
\ref{fig:rc}.
In mesh B, we find a current source whose current is proportional to the voltage $V_{b}$
(voltage controlled current source).
In mesh C, on the other hand, there is a voltage source that varies proportionally
with the current $I_c$ (current controlled voltage source).
The proportionality constants for these two cases are called, respectively,
transconductance (SI unit: Siemens - S) and transresistance (SI unit: Ohm - $\Omega$).
In Section~\ref{sec:analysis}, a theoretical analysis of the circuit is
presented. In Section~\ref{sec:simulation}, the circuit is analysed by
simulation, and the results are compared to the theoretical results obtained in
Section~\ref{sec:analysis}. The conclusions of this study are outlined in
Section~\ref{sec:conclusion}. In both the theoretical and simulation analysis the current directions for each branch are defined as in figure \ref{fig:rc}. The potencial $0$ was assigned to node $0$ (the one where the voltage is $V_{0}$), so this was the node used as a reference.

\begin{figure}[h] \centering
    \includegraphics[width=0.8\linewidth]{t1_meshes.pdf}
    \caption{Circuit analysed.}
    \label{fig:rc}
\end{figure}

