\section{Simulation Analysis}
\label{sec:simulation}

We decided to, once again, put the circuit's scheme down below, so as to make the interpretation of the following results easier (fig \ref{fig:op}).

\subsection{Operating Point Analysis}

Table~\ref{tab:op} shows the simulated operating point results for the circuit
under analysis. As can be seen, we obtained similar results to the ones calculated in the theoretical analysis section. This is proof that our theoretical analysis is, indeed, correct.
We noticed, however, that \emph{Octave} rounded the values of the circuit's parameters ($R_{1}, ..., R_{7}, V_{a}, I_{d}, K_{b}, K_{c}$), whereas \emph{Ngspice} operated with the same precision as the one provided initially (no rounding).
As such, one could expect to find slightly different results. However, this did not prove to be the case. Another important thing to note is that we had to insert a fictitional
voltage source, with value $0\hspace{1mm}V$, in series with $R_{6}$ and $R_{7}$, in order to properly introduce the current $I_{c}$ in $V_{c}$'s calculation, in the current dependent
voltage source. This was due to \emph{Ngspice}'s limitations, though. Thus,
the value $V(6.5)$ in the table, which is equal to $V(6)$, like we required it to.

\begin{figure}[h] \centering
  \includegraphics[height=0.33\textheight]{t1_meshes.pdf}
  \caption{Circuit analysed.}
  \label{fig:op}
\end{figure}

\begin{table}[h]
  \centering
  \begin{tabular}{|l|r|}
    \hline    
    {\bf Name} & {\bf Value [A or V]} \\ \hline
    \input{../sim/op_tab}
  \end{tabular}
  \caption{Operating point. A variable preceded by @ is of type {\em current}
    and expressed in Ampere; other variables are of type {\it voltage} and expressed in
    Volt.}
  \label{tab:op}
\end{table}
\newpage



