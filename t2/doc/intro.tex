\section{Introduction}
\label{sec:introduction}

% state the learning objective 
%The objective of this laboratory assignment is to study a circuit composed of
%four elementary meshes, where these can contain resistors, independent current or voltage sources
%and dependent current or voltage sources as seen in figure
%\ref{fig:rc}.

The objective of this laboratory assignment is to study a RC circuit containing a total of
seven nodes plus the ground (GND), seven resistors, two voltage sources, one independent the other
varies proportionally with the current $I_c$ - current controlled voltage source -, one dependent current
source proportional to the voltage $V_b$ - voltage controlled current source -, and finally a capacitor as 
seen in figure \ref{fig:rc}.
Futhermore, it is important to highlight that it is considered that the independent voltage source ($v_{s}$) supplies
in the following form:

\begin{equation}
    v_{s}\left( t\right) =\begin{cases}V_{s},t <0\\ \sin \left( 2\pi ft\right) ,t\geq 0\end{cases}
\end{equation}


%In mesh B, we find a current source whose current is proportional to the voltage $V_{b}$
%(voltage controlled current source).
%In mesh C, on the other hand, there is a voltage source that varies proportionally
%with the current $I_c$ (current controlled voltage source).
%The proportionality constants for these two cases are called, respectively,
%transconductance (SI unit: Siemens - S) and transresistance (SI unit: Ohm - $\Omega$).


In Section~\ref{sec:analysis}, a theoretical analysis of the circuit is
presented. We start with a nodal analysis with the goal of obtaining the voltage in every node, for  $t<0$
and from there getting the values of the current in every branch.
Hereinafter, we started the analysis for $t\geq0$. Firstly, to compute the natural solution ($v_{6n}(t)$)
using the initial conditions we calculated the equivalent resistance ($R_{eq}$), imposing $V_{x}$ = $V_{6}$ - $V_{8}$ which were
obtained in the previous step, as seen from the the capacitor. Then,
the forced solution was determined running the nodal analysis with phasor voltages and
replacing C with its impendance $Z_c$. The final solution is just the superposition of these two.
Finally, we determined the frequency responses of $v_{c}(f)$ = $v_{6}(f)$ - $v_{8}(f)$ and $v_{6}(f)$ for frequencies that ranges from
0.1Hz to 1MHz.

In Section~\ref{sec:simulation}, the circuit is analysed by
simulation, and the results are compared to the theoretical results obtained in
Section~\ref{sec:analysis}.

The conclusions of this study are outlined in
Section~\ref{sec:conclusion}.

In both the theoretical and simulation analysis some
current directions are defined as in figure \ref{fig:rc}.

\begin{figure}[h] \centering
    \includegraphics[width=\linewidth]{t2_RC.pdf}
    \caption{Circuit analysed.}
    \label{fig:rc}
\end{figure}

