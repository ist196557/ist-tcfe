\section{Simulation Analysis}
\label{sec:simulation}

\subsection{Operating point analysis (t$<$0)}

In the first part of the simulation we studied the circuit for t $<$ 0 using operating point analysis.
In order to do that, we assumed the same characteristics as described in subsection (\ref{subsec:analysis2})
with the addition that we had to add an additional voltage source of $0V$ in series with the resistor $R6$
working as some kind of "amperemeter" measuring the current $I{d}$ as it was necessary
since the current-controlled voltage source - $V_{d}$ - dependend on this value.
The results are shown in the following table (\ref{tab:ngs1}):

\begin{table}[h]
  \centering
  \begin{tabular}{|l|r|}
    \hline
    {\bf Name} & {\bf Value [V or A]} \\ \hline
    \input{../sim/t2_1_ngs_tab}
  \end{tabular}
  \caption{Values obtained for t$<$0.}
  \label{tab:ngs1}
\end{table}

Compared to the theoretical results, one notices that the results are similar
and only differ in digits with small order of magnitude. Aditionally, we tested the simulation
with different data values and the results were very similar to their theoretical counterparts as well.

\newpage

\subsection{Operating point analysis (t$=$0)}
In this section, we repeated the procedure described in subsection \ref{subsec:opA0}
 in \emph{Ngspice}, and obtained the following results for the node potentials, current $I_{x}$ and $R_eq$,
 shown in table (\ref{tab:ngs2}).

\begin{table}[h]
  \centering
  \begin{tabular}{|l|r|}
    \hline
    {\bf Name} & {\bf Value [V or A]} \\ \hline
    \input{../sim/t2_2_ngs_tab}
  \end{tabular}
  \caption{Values obtained for t$<$0.}
  \label{tab:ngs2}
\end{table}
The results obtained were similar to the ones calculated in subsection \ref{subsec:opA0}, with the only difference
being the extra precision that \emph{Ngspice} offers: where, for \emph{Octave}, $V(2)=0$,
 for \emph{Ngspice}, $V(2)\approx0$.
\newpage

\subsection{Sinusoidal analysis (t$>$0)}

\subsubsection{Natural solution}

The following graph shows the natural solution for node $6$, $v{6n}\left(t\right)$,
 according to the simulation that was run. Just like in the theoretical analysis, 
 the voltages in nodes $6$ and $8$ at $t=0$ were used as the initial conditions
  and the voltage source, $V{s}$, was set to $0$.
  
\begin{figure}[h] \centering
  \includegraphics[width=\linewidth]{t2_3_ngs.pdf}
  \caption{Natural solution.}
  \label{fig:nsg3}
\end{figure}

\newpage

\subsubsection{Total solution}

In this part we simulated the evolution in time of the total response on node 6 and the voltage source
stimulus as can be seen in the figure (\ref{fig:ngs4}):

\begin{figure}[h] \centering
  \includegraphics[width=\linewidth]{t2_4_ngs.pdf}
  \caption{Total solution.}
  \label{fig:ngs4}
\end{figure}

Comparing the results with the theoretical ones (obtained in \emph{Octave}), we can conclude that
they match the ones computed theoretically both in the phase and magnitude of the forced signal and
aswell in the evolution in time of the voltage source stimulus.

\newpage

\subsubsection{Frequency response}
%%%%%% Frequency response
The graphs below show the simulation results for the magnitude and phase in node 6,
 as well as in the source, for frequencies between $0.1 Hz$ and $1 MHz$. 
 The description of these graphics is similar to the one presented in the theoretical analysis.

\begin{figure}[h] \centering
  \includegraphics[width=\linewidth]{t2_5_magnitude.pdf}
  \caption{Magnitude [dB] vs frequency [Hz].}
  \label{fig:ngs5m}
\end{figure}

\newpage

\begin{figure}[h] \centering
  \includegraphics[width=\linewidth]{t2_5_phase.pdf}
  \caption{Phase [Degrees] vs frequency [Hz].}
  \label{fig:ngs5p}
\end{figure}