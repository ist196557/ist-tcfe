\newpage

\section{Conclusion}
\label{sec:conclusion}

In this laboratory assignment, the objective of analysing the circuit presented
in figure \ref{fig:rc} has been achieved. The analysis was performed both theoretically
using the Octave maths tool and by circuit simulation using the Ngspice tool.
Although the theoretical results differed slightly from the simulation results we did achieve good results overall.
The reason for this slight mismatch is the fact that the model for the diodes
in both were different, being the theoretical a simpler of the two.
In the end, we think that the goal of this laboratory was achieved. The building and optimization
of the circuit allowed us to further reflect about the behaviour of diodes and their relation with the other components already studied.


\begin{thebibliography}{9}

    \bibitem{spicenotes}
    Phyllis R. Nelson.
    \textit{\href{https://www.cpp.edu/~prnelson/courses/ece220/220-spice-notes.pdf}{Introduction to \emph{spice} source files}}.
    \\\texttt{https://www.cpp.edu/ $\tilde{}$ prnelson/courses/ece220/220-spice-notes.pdf}


    \bibitem{octave}
    John W. Eaton, David Bateman, Søren Hauberg, Rik Wehbring.
    \textit{\href{https://octave.org/octave.pdf}{GNU Octave - Free Your Numbers}}.
    \\\texttt{https://octave.org/octave.pdf}

    \bibitem{libreoffice}
    LibreOffice Documentation Team.
    \textit{\href{https://documentation.libreoffice.org/assets/Uploads/Documentation/en/GS7.0/GS70-GettingStarted.pdf}{Getting Started Guide}}.
    \\\texttt{https://documentation.libreoffice.org/assets/Uploads/Documentation/en/
        GS7.0/GS70-GettingStarted.pdf}

    \bibitem{analysis}
    Mesh analysis and node analysis notes.
    \textit{\href{https://moodle.fct.unl.pt/pluginfile.php/167552/mod_resource/content/0/NotasTecnicas/Apontamento_tecnico_3_ebf.pdf}{Princípios fundamentais na análise de circuitos electrónicos}}.
    \\\texttt{https://moodle.fct.unl.pt/pluginfile.php/167552/mod$\_$resource/content/
        0/NotasTecnicas/Apontamento$\_$tecnico$\_$3$\_$ebf.pdf}

\end{thebibliography}
